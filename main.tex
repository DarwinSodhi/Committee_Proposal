\documentclass{article}
\usepackage[utf8]{inputenc}

\title{Committee Proposal}
\author{Darwin Sodhi}
\date{February 2020}
\documentclass[12pt,letter]{article}
\usepackage[top=1.00in, bottom=1.0in, left=1.1in, right=1.1in]{geometry}
\renewcommand{\baselinestretch}{1.1}
\usepackage{graphicx}
\usepackage{natbib}
\usepackage{amsmath}
\usepackage{siunitx}
\usepackage{adjustbox} 


\def\labelitemi{--}
\parindent=0pt


\begin{document}
\maketitle

\section{Introduction}

Ecological communities are ripe with disease, these diseases shape species populations, thus understanding disease dynamics is a top priority for community ecologists and epidemiologists alike. A key issue in disease research is to understand how pathogens can alter both inter- and intraspecific interactions, such as competition and predation. Tackling this issue would allow ecologists to better understand how species can coexist in the same environment even though they appear to occupy similar niches (\citep{Freckleton2006}). Further, understanding the role pathogens play in coexistence dynamics will allow researchers to improve our understanding of changing communities and help in mitigating the negative impacts of climate change.  

\paragraph{}Plant pathogens (a mirco-organism that causes disease) can affect species coexistence by causing non-random differences in growth and mortality, which can lead to rapid population declines and large shifts in the structure of plant communities (\citep{Gilbert2002}). For example, if pathogens disproportionately  reduce species’ fitness as it becomes more common (termed ‘negative density-dependence or NDD hence fourth), this can help reduce dominance by any one species and increase diversity (\citep{Chesson2000}). In natural plant communities, pathogens may be especially important in promoting diversity via increasing seedling mortality closer to conspecific adult trees, thereby reducing local dominance (\citep{HilleRisLambers2012})—the Janzen- Connell hypothesis. The Janzen-Connell hypothesis has been widely studied in the tropics and temperate regions (cite Terborgh 2012 and Comita et al 2014) but its importance in structuring plant communities appears highly specie and site specific, making generalizations difficult to date.  

\paragraph{}It has been almost 50 years since the Janzen-Connell hypothesis was first proposed, since then the original paper have been cited almost 2000 times (\citep{Comita2014}). Testing of the Janzen-Connell hypothesis has often focused on the early stages of tree seedling development. A recent hypothesis in the Janzen-Connell literature has postulated that the Janzen-Connell effect varies with latitude and this could be explained by differences in pathogen efficiency (Hille Ris Lambers, Clark Beckage 2002). Differences within and between species in the Janzen-Connell effect has also been shown (cite Harms 2000 and Comita Science 2010), however differences in pathogen load between species is often overlooked when trying to explain this phenomenon (Nathan Muller-Landau 2000; Comita et al. 2007). 

\paragraph{}Through my PhD I want to further our understanding of species coexistence as it occurs in nature. In order to do this I will firstly, conduct a long term field experiment (of tree seed and seedling fitness) across a latitudinal gradient in British Columbia to test the presence of the Janzen-Connell hypothesis. Secondly, I will conduct a green house experiment where I test the density dependence and distance mortality of seedlings for the most abundant tree species in my field sites. Finally, I test the assumption that related species share pathogens in an agricultural setting using winegrape (Vitis vinifera) pathogens as a model specie. 

\section{Field Experiment}
The Janzen-Connell hypothesis postulates that plant performance during early life stages (i.e. seed ger- mination, seedling and sapling recruitment, growth and survival) is lower close to versus far from parent conspecific trees and varies with high versus low conspecific adult densities (\citep{Comita2014}). The role of pathogens is key to this hypothesis, because there is an underlying assumption that greater conspecific densities or shorter distances to conspecifics will yield higher disease pressure on seedlings. This underlying assumption is rarely tested and often only inferred from the data collected in the field. I will test for disease loads of seeds and seedings in the field. Another underlying assumption of the Janzen-Connell hypothesis is that patterns will be stronger in the tropics compared to the temperate ecosystems as pathogen pressure has been hypothesized to be greater in the tropics compared to the temperate forests (\citep{Comita2014}). I will test the Janzen-Connell hypothesis across a latitudinal gradient in multiple sites across British Columbia. 

\subsection{Methods}
In my proposed research, I will investigate how multiple mechanisms (niche partitioning and pathogen loads) contribute to the assembly of communities across a latitude gradient within British Columbia based on the Janzen-Connell Hypothesis. This gradient will allow me to better understand patterns across scales (within a site, between sites and across all sites) and environmental gradients. This is important because different mechanisms could be operating at different scales, and dominant mechanisms might by context specific. Suitable forest sites have already been identified: Manning park (Hope, BC), Alex Fraser Research Forest (Big Lake, BC), and various locations in Smithers, BC. I will be setting up 50 plots (25 meters by 25 meters) at each site for a total of 150 plots. At each site every adult tree (trees that are larger than 1.37 meters) will be identified, tagged and geo-referenced. I will then tag every seedling (trees smaller than 1.37 meters) within a site, measuring its height, distance to nearest conspecific adult tree. Across multiple years, we will come back to each plot and measure change in growth of the seedling which will be used as proxy for plant performance. #(How will you deal with seedlings that die?)

\subsection{Predictions}
I predict that tree seedlings that are further away from conspecific will have higher growth rates compared to trees that are closer to adult conspecifics. In addition, adult conspecific densities will be expected to also alter growth rates of seedlings. A seedling that has high amounts of adult conspecific densities around would also be expected to have lower growth rates compared to seedlings with lower conspecific densities around. Both of these conditions are central to the Janzen-Connell hypothesis and this assumes shared generalist pathogens between seedlings and adult conspecifics (\citep{Janzen1970}, \citep{Connell1978}). While specific pathogen loads will be directly measured, which requires obtaining samples from the seedlings and sending them for DNA analysis. Based on previous work, our Manning Park site should show the strongest patterns of the Janzen-Connell hypothesis because this site is the one with the lowest latitude (Hille Ris Lambers, Clark & Beckage 2002). This could translate to lower performance of seedling trees at larger distances to adult conspecific (pathogens are able to travel further) or lower performance of seedling trees at smaller distances (pathogens are more efficient). To truly unpack this we will need to compare plant performance across all of our sites. We would expect that Smithers (our most northern site) and Alex Fraser (our intermediate site) to show different patterns based on their latitude. 

\section{common garden}
Field experiments are messy because environmental factors that mediate species fitness and interactions, making the relative importance of disease more difficult to measure. A common garden that controls environmental factors and isolate the impacts from disease. I will use a common garden experiment to test if densities of conspecifics affect seedling growth and mortality (NDD) and the scale (density and distance) at which this is important. I will also set up a ‘mother tree’ inside of the greenhouse to examine the role of distance in conferring disease to offspring and I will measure pathogen loads to test the underlying assumption that species closer to conspecifics will have a larger pathogen loads. 

\subsection{Methods}
For this common garden, I will focus on the four most abundant tree species across our sites. To test the assumption that species performance is worse with higher conspecific densities, I will plant species individually which will be used as the null model, I will then alter conspecific densities across pots ranging from 1-4. To test the scale at which this matters we can alter conspecifics across a variety of distances ranging from 5 center meters to 20 center meters. To measure performance, growth rate will be measured by measuring height every week. To quantify pathogen load, I will plant tree seedlings alone, the same tree species with a conspecific and finally, the same species with a hetrospecific. I will then send leaves from this experiment to be analyzed to quantify pathogen load for each replicate. I will have four replicates per plant for each for a total of 16 groups per experiment. All plants will be plotted in the same soil and be given the same amount of water per week.

\subsection{Predictions}
My predictions are that the tree seedlings planted alone will have the largest growth rates. Tree seedlings planted with conspecifics will have lower growth rates with a higher number of the conspecific planted together. For the distance dependency experiment, tree seedlings planted alone would be expected to have the largest growth rates. Further, the closer a conspecific is planted the lower the expected growth rate. Both of these predictions follow closely with expected results from the Janzen-Connell hypothesis if pathogens are indeed shared. For the pathogen load experiment, we would expects that when seedlings are planted with conspecifics they have a higher pathogen load compared to when seedlings are planted alone or with a hetrospecific species. This will allow us to specifically test another underlying assumption of the Janzen-Connell hypothesis that species closer to conspecifics will have higher pathogen loads. 

\section{Disease impacts in Agricultral Environments}
One the the largest assumptions regarding the Janzen-Connell hypothesis is that specialized pathogens will cause seedling mortality close to conspecific adult trees. This assumption has rarely been tested in both pathogens and agricultural settings. To investigate if species that are closely related do indeed share pathogens, I looked at the phylogenetic structure of winegrape (\textit{Vitis vinifera}) pathogens. Phylogenetics is particularly useful to tackle this because species with close evolutionary relationships are thought to be similar in physiological, immunological, and life history traits that influence infection risk (\citep{Davies2008}, \citep{Gilbert2007}). 

\subsection{Methods}
We used three online resources (www.webofscience, www.cabi.org and www.scalenet.info) to build a preliminary list of the major pathogens affecting Winegrapes and the agricultural host species they also infect. We queried each database using the following search terms: "wine grape pathogens", "Vitis vinifera pathogens", and "wine grape pathogen impacts", and recorded data for all species for which we could obtain published data on \textit{Vitis vinifera} infection rates and yield losses. This returned a list of forty-nine (see supplementary list) pathogens, which we separated into four broad categories: fungal, bacterial, viral and pest. We cleaned taxonomic nomenclature, resolving synonyms, and obtained a list of recorded hosts from the following databases: Description of plant Viruses (http://www.dpvweb.net/), Nematode-Plant Expert Information System (http://nemaplex.ucdavis.edu/), and  U.S. National Fungus Collections Fungal Database (https://nt.ars-grin.gov/fungaldatabases). For each pathogen, we obtained georeferenced locations and native geographic distribution from cabi.org.

\paragraph{}To characterise the phylogenetic distribution of winegrape pathogens on their known agricultural hosts, we first used the list of agricultural host species (n = 944) from (\citep{Milla2018}) subset our host list, returning the list of agricultural hosts infected by each winegrape pathogen. Second, we pruned the more inclusive age-calibrated phylogenetic tree (\citep{Zanne2014}) to our list of agricultural host species. Because some hosts were only identified to genus, we performed two sets of analyses, one that assumed such pathogens infected all agricultural species in that genus, and another assuming only a single arbitrary host species within the genus was infected (ensuring that this species was represented in the Zanne et al.2014 phylogeny\citep{Zanne2014}). By assuming all host species in a genus are infected, we likely inflate taxonomic clustering of pathogen host range;  however, using only one single arbitrary specie would likely reduce taxonomic clustering. By conducting the two analyses we thus bracket the possible extremes.

\paragraph{}We used the PICANTE (\citep{PICANTE} package in R to quantify the phylogenetic clustering of host species infected by each pathogen. Using the subset phylogeny from Zanne et al.\citep{Zanne2014}, we calculated the mean pairwise distances (MPD) between all agricultural hosts for each pathogen, the mean nearest taxon distance (MNTD), which describes the minimum pairwise distance separating hosts for each pathogen, and the mean distance to \textit{Vitis vinifera} from each non \textit{V. vinifera} host, which we refer to as the focal phylogenetic distance (FPD) . Standard effect sizes (SES) were calculated for each measure using the following equation, assuming a null model of no phylogenetic structure (option: tip swap in PICANTE), and 999 replicates.

\paragraph{}We explored variation in phylogenetic structure across pathogens using regression models and Bayesian analyses in the Stan probabilistic programming language (Stan Development Team:. http://mc-stan.org). First, we tested for broad differences in the phylogenetic dispersion of infected hosts across taxonomic groupings, separating pathogens into five categories: fungal, bacterial, viral, pest and nematode. Second, we evaluated differences in host phylogenetic dispersion between specialist (pathogens infecting only a single genus) and generalist (pathogens infecting multiple genera) pathogens. Finally, we tested for trait differences, using body size information for nematodes and pests obtained from Nematode-Plant Expert Information System (http://nemaplex.ucdavis.edu/) and Centre for Agriculture and Bioscience International database (www.cabi.org). Equivalent data for body size for bacteria and fungi are not available. All statistical analyses were carried out using the R statistical software (\citep{R}
) and RStanArm (\citep{RSTANARM}). To assess pathogen impact on (\textit{Vitis vinifera}, we extracted data on yield loss from the Centre for Agriculture and Bioscience International database (cabi.org) and the peer-reviewed literature (see above). When ranges or multiple values were reported, we took the maximum. We analyzed our data using Bayesian models using RStanArm. Stan allows effective Markov Chain Monte Carlo sampling via a No‐U‐Turn Hamiltonian Monte Carlo approach as stated in Flynn and Wolkovich 2019 (\citep{Flynn2018a}). For both Bayesian models normal priors were chosen with a mean of zero and a standard deviation of 3 (increasing the priors three‐fold did not change the model results). We ran four chains simultaneously, with 1000 warm‐up iterations followed by 1000 sampling iterations.


\subsection{Results}
Our phylogenetic analysis on pathogen host range using MPD and MNTD metrics, showed that for most pathogens their pathogen host ranges are strongly clustered. Within the pathogen host range there are a select group of species that show differing patterns for MPD and MNTD metrics however, these pattern are statistically weak. Our bayesian models showed that pathogen type (Whether a pathogen is classified as a fungal, bacterial, nematode, viral or pest) and pathogen category (whether a pathogen is generalist or specialist) were important predictors for pathogen host ranges. Finally our bayesian model observing impact on winegrapes showed that when pathogen host range is clustered around winegrapes (FPD metric), the impact on yield loss on winegrapes is higher compared to if the pathogen host range is dispersed. When testing the same model with MPD as the predictor we observe the same result although, model cross validation showed that the model with MPD as the predictor was better suited in predicting winegrape impact compared to FPD.

\subsection{Discussion}
Our results broadly show that species that are closely related would be expected to share more pathogens. In addition, the more clustered the pathogen, the higher impact it would expect to have on its host. Both of these results are in agreement with assumptions of the Janzen-Connell hypothesis. Our results also points to including phylogenetics into studies that test the Janzen-Connell (\citep{Liu2012}) as if tree species in a community are clustered then specialist pathogens may be able to host jump to other tree seedlings, this could explain why in temperate ecosystems the results of the Janzen-Connell hypothesis have been mixed (\citep{Hyatt2003}). Another interesting aspect to test would be if across a latitude, the pathogens become more dispersed or clustered as you move close to the equator. Efficiency could be determined by the number of hosts a pathogen is found on or the level to which a pathogen infects the hosts in terms of fitness costs. 

\paragraph{}Our results highlight that pathogens play an imperative role in shaping communities (\citep{Janzen1970}, \citep{Connell1978}, \citep{Bush1997}). Although the exact effect will have to be evaluated in natural ecosystems. Pathogens may reduce the fitness of their hosts which in turn can promote coexistence among species that normally would be expected to competitively exclude each other (\citep{Spear2018}). An important consequence of our work is that the assumption in the Janzen-Connell that species that are phylogenetically similar share similar pathogens. The logical next step would be to observe the impact of pathogens across its host breathe as this will further our understanding on species coexistence in nature. 

\bibliographystyle{unsrt}
\bibliography{Winegrapepaper.bib}

\end{document}
